\documentclass{cernatsnote}
\usepackage[colorinlistoftodos]{todonotes}
\usepackage{placeins}

\title{CERN ATS Note title}
\author{
	Author Name \; \\		
	CERN, CH-1211 Geneva, Switzerland
}
\email{author.email@cern.ch}
\date{\today}

\begin{document}
\maketitle

\begin{abstract}
This document shows how to calculate the path-length of rectangular bending magnets in a beam line. The path-length depends on the pole-face angles, i.e. how the magnet is positioned in the line. The majority of bending magnets are installed with identical pole-face angles at the start and the end, but in certain cases the pole-face angles are different e.g. in the CERN PS BOOSTER BTP and BTY extraction lines, the BHZ10 magnet have a special positioning in order not to perturb the optics of any of the lines unfavorably.
 The path-length correspond to the s-parameter in MADX, and must be calculated precisely, in order to get a correct survey, which need to be correct to the 10 micron level.
\end{abstract}
\\ \\ \\ 

\begingroup
\color{black}
\tableofcontents
\endgroup

\pagebreak

\section{Introduction}


\bibliography{references}
\bibliographystyle{plain}

\end{document}